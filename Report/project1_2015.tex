\documentclass[11pt]{article}
\usepackage{verbatim}
\usepackage{listings}
\usepackage{graphicx}
\usepackage{a4wide}
\usepackage{color}
\usepackage{amsmath}
\usepackage{amssymb}
\usepackage[dvips]{epsfig}
\usepackage[T1]{fontenc}
\usepackage{cite} % [2,3,4] --> [2--4]
\usepackage{shadow}
\usepackage{hyperref}
\usepackage{physics}
\usepackage{url}
\usepackage{tikz}
\usepackage{subcaption}


\usetikzlibrary{arrows, shapes}

\setcounter{tocdepth}{2}

\lstset{language=c++}
\lstset{alsolanguage=[90]Fortran}
\lstset{basicstyle=\small}
\lstset{backgroundcolor=\color{white}}
\lstset{frame=single}
\lstset{stringstyle=\ttfamily}
\lstset{keywordstyle=\color{red}\bfseries}
\lstset{commentstyle=\itshape\color{blue}}
\lstset{showspaces=false}
\lstset{showstringspaces=false}
\lstset{showtabs=false}
\lstset{breaklines}

\title{ Computational Physics II \\ FYS-4411 }
\author{Gullik Vetvik Killie\\
		Håkon S.B. Mørk\\
		Jose Emeilio Ruiz Navarro
		}

\begin{document}

\maketitle

\abstract{Fill in abstract}

\section{Introduction}

\section{Methods}
	\subsection{Derivation of local energies}
		\subsubsection{Helium: Simple trialfunction}
		The simple version of the trial function is only dependant on one parameter \( \alpha \) and does not take into account interaction between the two electrons, it is of the form 
		\[ \Psi_T (\vb{r_1}, \vb{r_2}) = \exp{ -\alpha (r_1 + r_2) } \]

		\begin{align}
		E_L &= \frac{1}{ \Psi_T } \hat{H} \Psi_T 
		\\
		&=	\exp{ \alpha (r_1 + r_2) } \left( -\frac{nabla_1^2}{ 2 } - \frac{nabla_2^2}{ 2 } - \frac{2}{r_1}  -\frac{2}{r_2} +  \frac{1}{r_{12}} \right)   \exp{ -\alpha (r_1 + r_2) }
		\end{align} 
	
\section{Results and discussion}

\section{Conclusions and perspectives}

\appendix


	The local energy for the simple trialfunct

	\[
	Z \left(- \frac{1}{r_{2}} - \frac{1}{r_{1}}\right) - \alpha^{2} + \alpha \left(\frac{1}{r_{2}} + \frac{1}{r_{1}}\right) + \frac{1}{r_{12}}
	\]


\end{document}



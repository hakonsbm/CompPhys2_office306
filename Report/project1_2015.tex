\documentclass[11pt]{article}
\usepackage{verbatim}
\usepackage{listings}
\usepackage{graphicx}
\usepackage{a4wide}
\usepackage{color}
\usepackage{amsmath}
\usepackage{amssymb}
\usepackage[dvips]{epsfig}
\usepackage[T1]{fontenc}
\usepackage{cite} % [2,3,4] --> [2--4]
\usepackage{shadow}
\usepackage{hyperref}
\usepackage{physics}
\usepackage{url}
\usepackage{tikz}
\usepackage{subcaption}


\usetikzlibrary{arrows, shapes}

\setcounter{tocdepth}{2}

\lstset{language=c++}
\lstset{alsolanguage=[90]Fortran}
\lstset{basicstyle=\small}
\lstset{backgroundcolor=\color{white}}
\lstset{frame=single}
\lstset{stringstyle=\ttfamily}
\lstset{keywordstyle=\color{red}\bfseries}
\lstset{commentstyle=\itshape\color{blue}}
\lstset{showspaces=false}
\lstset{showstringspaces=false}
\lstset{showtabs=false}
\lstset{breaklines}

\title{ Computational Physics II \\ FYS-4411 }
\author{Gullik Vetvik Killie\\
		Håkon S.B. Mørk\\
		Jose Emeilio Ruiz Navarro
		}

\begin{document}

\maketitle

\abstract{Fill in abstract}

\section{Introduction}

\section{Methods}
	\subsection{Monte Carlo of the Helium Atom}
		In a quantum mechanical system the energy is given by the expectation value of the Hamiltonian. 

		\begin{align}
			E[H] = \bra{\Psi}
		\end{align}


	\subsection{Derivation of local energies}
		The local energy of is dependant on the Hamiltonian and the wavefunction describing the system, the Hamiltonian incorporates both a kinetic energy part given by \( \frac{\nabla_i^2}{2} \) for each particle
		and a potential energy part given by \(\frac{Z}{r_i}\) and \(\frac{1}{r_{ij}}\), where \(Z\) is the charge of the center, \(r_i\) is the distance for electron \(i\) to the atom center and \(r_{ij}\) is the distance between electron \(l\) and \(m\). Then the local energy is given by the following:

		\begin{align}
			E_L &= \sum_{i,i<j}{\frac{1}{ \Psi_T(\vb{r_i} , \vb{r_{ij}}) } \hat{H} \Psi_T(\vb{r_i} , \vb{r_{ij}})}
			\\
			&=	\sum_{i,i<j}\frac{1}{ \Psi_T(\vb{r_i} , \vb{r_{ij}}) } \left( - \frac{\nabla_i^2}{2} -\frac{Z}{r_i}  -  \frac{Z}{r_j} +  \frac{1}{r_{ij} }  \right) \Psi_T(\vb{r_i} , \vb{r_{ij}})
			\\
			&= \sum_{i,i<j}{-\frac{1}{2\Psi_T} \left(\nabla_i^2 \Psi_T  \right)  -\frac{Z}{r_i}  -  \frac{Z}{r_j} +  \frac{1}{r_{ij} }}
		\end{align}

		Let us change derivation variables:

		\begin{align}
			-\frac{1}{2\Psi_T} \left(\nabla_i^2 \Psi_T  \right) &= \sum_{m=1}^{3}{-\frac{1}{2\Psi_T} \left( \pdv[2]{\Psi_T}{x_m} \right)_i}
			\\
			&= \sum_{m=1}^{3}{-\frac{1}{2\Psi_T} \left( \pdv{}{x_m} \left( \pdv{\Psi_T}{r_i}\pdv{r_i}{x_m} \right) \right)_i}
			\intertext{Since \(r_i = \left( x_1^2 + x_2^2 + x_3^2 \right)^{1/2}\) then \( \pdv{r_i}{x_m} = \pdv{\left( x_1^2 + x_2^2 + x_3^2 \right)^{1/2}}{x_m} =\frac{x_m}{r_i} \)}
			&= \sum_{m=1}^{3}{-\frac{1}{2\Psi_T} \left( \pdv{}{x_m} \left( \pdv{\Psi_T}{r_i}\frac{x_m}{r_i} \right) \right)_i}
			\\
			&= \sum_{m=1}^{3}{-\frac{1}{2\Psi_T} \left( \pdv{\Psi_T}{x_m}{r_i}\frac{x_m}{r_i} + \pdv{\Psi_T}{r_i} \pdv{}{x_m} \left(\frac{x_m}{r_i} \right) \right)_i}
			\intertext{ The term \( \pdv{}{x_m} \left(\frac{x_m}{r_i} \right) \) becomes for the different values for \(m\),  \(\pdv{}{x_1}  \left( \frac{x_1}{\left( x_1^2 + x_2^2 + x_3^2 \right)^{1/2}} \right) = \frac{x_2^2 + x_3^2}{r_i^3}\) so all the values for \(m\) term it should sum up to \( \frac{ 2 (x_1^2 + x_2^2 + x_3^2) }{ r_i^3 } \) }
			&= -\frac{1}{2\Psi_T} \left( \pdv[2]{\Psi_T}{r_i}\frac{x_1^2 + x_2^2 + x_3^2}{r^2_i} + \pdv{\Psi_T}{r_i} \frac{ 2 (x_1^2 + x_2^2 + x_3^2) }{ r_i^3 } \right)_i
			\\
			&= -\frac{1}{2\Psi_T} \left( \pdv[2]{\Psi_T}{r_i} + \pdv{\Psi_T}{r_i} \frac{ 2 }{ r_i } \right)
		\end{align}
		Then the local energy becomes:
		\begin{align}
			E_L = \sum_{i,i<j}{  -\frac{1}{2\Psi_T} \left( \pdv[2]{\Psi_T}{r_i} + \pdv{\Psi_T}{r_i} \frac{ 2 }{ r_i } \right)  -\frac{Z}{r_i}  -  \frac{Z}{r_j} +  \frac{1}{r_{ij} }} \label{eq:localEnergy}
		\end{align}


		\subsubsection{Helium: Simple trialfunction}
		The simple version of the trial function is only dependant on one parameter \( \alpha \) and does not take into account interaction between the two electrons, it is of the form 
		\[ \Psi_T (\vb{r_1}, \vb{r_2}) = \exp{ -\alpha (r_1 + r_2) } \]Let us set this trialfunction into the equation for the local energy \eqref{eq:localEnergy}.
		\begin{align}
			E_L &= \sum_{i,i<j}{  -\frac{1}{2\Psi_T} \left( \pdv[2]{e^{-\alpha (r_i + r_j)}}{r_i} + \pdv{e^{-\alpha (r_i + r_j)}}{r_i} \frac{ 2 }{ r_i } \right)  -\frac{Z}{r_i}  -  \frac{Z}{r_j} +  \frac{1}{r_{ij} }} 
			\\
			E_L &= -\frac{1}{2\Psi_T} \sum_{i=1}^2{ \left( \alpha^2 -\alpha \frac{ 2 }{ r_i } \right) \Psi_T  -\frac{Z}{r_i} +  \frac{1}{r_{ij} } }
			\\
			E_L &= -\alpha^2 + (\alpha-Z) \left( \frac{1}{r_1} + \frac{1}{r_2} \right) + \frac{1}{r_{12}} 
		\end{align}




\section{Results and discussion}

\section{Conclusions and perspectives}

\appendix


	The local energy for the simple trialfunct

	\[
	Z \left(- \frac{1}{r_{2}} - \frac{1}{r_{1}}\right) - \alpha^{2} + \alpha \left(\frac{1}{r_{2}} + \frac{1}{r_{1}}\right) + \frac{1}{r_{12}}
	\]


\end{document}



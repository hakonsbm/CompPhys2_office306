%Contains a broad overview over the monte carlo method

\subsection{Monte Carlo method with simple Metropolis sampling}
		In a quantum mechanical system the energy is given by the expectation value of the Hamiltonian, let \(\Psi_T\) be a proposal for a wavefunction that can describe the system.

		\begin{align}
			E[\hat{H}] = \expval{\hat{H}}{\Psi_T} = \frac{\int{d\vb{R} \Psi_T^*(\vb{R})\hat{H} \Psi_T(\vb{R})  }}{ \int{d\vb{R} \Psi_T^*(\vb{R}) \Psi_T(\vb{R}) }}
		\end{align}

		Let us introduce a local energy:

		\begin{align}
			E_L(\hat{H}) &= \frac{1}{ \Psi_T(\vb{R}) } \hat{H} \Psi_T(\vb{R}))
		\end{align}

		\begin{align}
			E[\hat{H}] &= \frac{\int{d\vb{R} \Psi_T^*(\vb{R}) \Psi_T(\vb{R}) E_L(\vb{R}))  }}{ \int{d\vb{R} \Psi_T^*(\vb{R}) \Psi_T(\vb{R}) }}
			\intertext{Since the denumerator is a scalar constant after integrating it we can put it inside the integral in the numerator}
			E[\hat{H}] &= \int{d\vb{R} \frac{\Psi_T^*(\vb{R}) \Psi_T(\vb{R})  }{\int{d\vb{R'} \Psi_T^*(\vb{R'}) \Psi_T(\vb{R'})}}  E_L(\vb{R})  }
			\\
			E[\hat{H}] &= \int{d\vb{R} P(\vb{R}) E_L(\vb{R}) }
		\end{align}

		This probability function with \(P(\vb{R})\) as the pdf, and we can use monte carlo integration to solve the integral.

		\begin{enumerate}
			\item Initialise system. Give particles a random position and decide how many Monte Carlo Cycles to run.
			\item Start Monte Carlo Calculations
				\begin{enumerate}
					\item Propose a move of the particles according to an algorithm, for example \newline \( \vb{R_{new}} = \vb{R_{old}} + \delta * r \), where \(r\) is a random number in \([0,1]\)
					\item Accept or reject move according to \( P(\vb{R_{new}})/ P(\vb{R_{old}}) \ge r \), where r is a new number. Update position values if accepted.
					\item Calculate energy for this cycle.
				\end{enumerate}
		\end{enumerate}
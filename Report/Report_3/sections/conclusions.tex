We see that the Variational Monte Carlo method gives good values, both for small systems like Helium and Beryllium, and also for larger systems like Neon, as long as we deal with the interaction between the particles using the Jastrow factor. We have implemented MPI, making it possible to split up the Monte Carlo simulation in pieces and distribute it to multiple processors. This reduces the time to run simulations considerably. With a larger system we also needed a better way to handle the Slater Determinant. Using Slater Type Orbitals we get the best values, but using this makes the calculations dependent on two variables. This makes us able to use the bisection method to find the optimal beta value. 

Taking a look at \ref{tab:EnergyAlphaBetaReference} it is apparent that the results are quite precise. In the case of Helium the experimental value is less than $0.5\%$ higher than the computational one. For Beryllium the experimental value is just under $2\%$ higher than the reference value, and for Neon, the difference is even less at just under $1\%$. 

\textbf{CHARGE DENSITY CONCLUSIONS}. 

These results are very good considering the amount of resources that went into calculating them: a simple standard PC and a handful of hours were enough. Considering that a relatively simple and very well tested method was used to obtain these results it is very easy to see why VMC is so popular for solving quite complicated problems like calculating the ground state energies of non-trivial quantum mechanical systems.

Looking ahead, to improve the Variational Monte Carlo program we could calculate more analytical functions for the Slater Determinant and for the derivatives of the GTO functions. These parts are used a lot during the program and switching from numerical to analytical solution would speed things up. Further, as the Gaussian Type Orbitals gave slightly dissapointing energies, we could implement another, larger basis set than the 3-21G basis, such as the 6-311G basis. This would give the GTO functions a shape that more closely resembles Slatere Type Orbitals and therefore give better energies.  

\subsection{Critique on the exercise}
	The exercise is well made and has a good learning curve during the three projects. We feel that we have gained insight into calculation of physical properties using Monte Carlo methods. There are however a couple things that could be better. The charge density and GTOs could be better explained and feedback on project 1 and 2 could have come sooner. 
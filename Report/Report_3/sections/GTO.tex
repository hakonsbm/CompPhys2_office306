
\subsection{Gaussian type orbitals}

As proposed in a paper by Boys in 1950 \parencite{Boys_1950},
Gaussian Type Orbitals, or GTOs, can be used in electronic structure
theory, and it is today common to use them, especially in computational
chemistry. Most importantly for us they make us able to limit the
variational calculations of complex atoms to one variational variable,
thus reducing the number of variational possibilities significantly.
This is obviously a great advantage, because as the number of particles
in our calculations grow, the energy takes considerably longer time
to compute for each combination of variables. 


\subsubsection{Using GTOs to replace the Slater type orbitals}

To replace our current orbitals we need to calculate primitive GTOs
and contract them to contracted GTOs. A contracted GTO, $\phi$,
is defined as
\[
\phi\left(x,y,z\right)=\sum_{i}N_{i}\chi_{i}\left(x,y,z\right),
\]
where $\chi_{i}$ is a primitive GTO and $N_{i}$ is a normalization
constant. The primitive GTO is defined as
\[
\chi_{i}\left(x,y,z\right)=c_{i}x^{m}y^{n}z^{o}e^{-\alpha_{i}R^{2}}.
\]
Here $x$, $y$, and $z$ are Cartesian coordinates representing the
distance to a nucleus, and $R^{2}$=$x^{2}+y^{2}+z^{2}$. The quantum
numbers $m$, $n$ and $o$ depend on the angular momentum of the
orbital. The numbers $c_{i}$ and $\alpha_{i}$ are variational parameters,
which can be fetched from an online library, namely EMSL \parencite{Binkley_1980}\parencite{EMSL}.

When combining contracted primitives into orbitals that are to mimic
and replace the Slater type orbitals additional constants are needed,
one for each contracted GTO. These are calculated using Hartree-Fock
calculations. For Helium, Beryllium and Neon these constants are given
in table \ref{tab:He_Be_Ne_K}. 


As an example, for Helium, we find the parameters to be

\[
\begin{array}{ccc}
 & \alpha_{i} & \mbox{c}_{i}\\
\chi_{1} & 13.62670 & 0.175230\\
\chi_{2} & 1.999350 & 0.893483\\
\chi_{3} & 0.382993 & 1.000000
\end{array}
\]

Thus the Gaussian orbital becomes
\begin{eqnarray*}
\phi\left(x,y,z\right) & = & 0.4579\times\left(\left(\frac{2\times13.62670}{\pi}\right)^{3/4}0.17523e^{-13.62670\times R^{2}}+\left(\frac{2\times1.99935}{\pi}\right)^{3/4}0.893483\times e^{-1.99935\times R^{2}}\right)\\
 &  & +0.6573\times\left(\left(\frac{2\times0.382993}{\pi}\right)^{3/4}1.0e^{-0.382993\times R^{2}}\right)
\end{eqnarray*}


\begin{table}
	\begin{centering}
	\begin{subtable}{1.0\linewidth}
		\begin{centering}
		\begin{tabular}{|c|}
			\hline 
			1s\tabularnewline
			\hline 
			0.4579\tabularnewline
			\hline 
			0.6573\tabularnewline
			\hline 
		\end{tabular}
		\par\end{centering}
	\end{subtable}
	\subcaption{Helium}

	\begin{subtable}{1.0\linewidth}
		\begin{centering}
		\begin{tabular}{|c|c|}
			\hline 
			1s & 2s\tabularnewline
			\hline 
			-9.9281e-01 & -2.1571e-01\tabularnewline
			\hline 
			-7.6425e-02 & 2.2934e-01\tabularnewline
			\hline 
			2.8727e-02 & 8.2235e-01\tabularnewline
			\hline 
			1.2898e-16 & 5.1721e-16\tabularnewline
			\hline 
			-2.3257e-19 & 4.5670e-18\tabularnewline
			\hline 
			5.6097e-19 & -1.1040e-17\tabularnewline
			\hline 
			1.2016e-16  & 8.5306e-16\tabularnewline
			\hline 
			-4.6874e-19 & 7.0721e-18\tabularnewline
			\hline 
			1.1319e-18 & -1.7060e-17\tabularnewline
			\hline 
		\end{tabular}
		\par\end{centering}
	\end{subtable}
	\subcaption{Beryllium}

	\par
	\begin{subtable}{1.0\linewidth}
		\begin{centering}
		\begin{tabular}{|c|c|c|c|c|}
			\hline 
			1s & 2s & $2p_{x}$  & $2p_{y}$ & $2p_{z}$\tabularnewline
			\hline 
			-9.8077e-01 & -2.6062e-01 & 1.1596e-16 & -8.3716e-18 & -1.9554e-17\tabularnewline
			\hline 
			-9.3714e-02 & 2.5858e-01 & -2.0106e-16 & -9.7173e-17 & -7.3738e-17\tabularnewline
			\hline 
			2.2863e-02 & 8.1619e-01 & -3.2361e-16 & 1.3237e-16 & 1.5789e-16\tabularnewline
			\hline 
			-9.9519e-19  & -5.6186e-18 & 2.7155e-02 & -4.0320e-01 & 3.9171e-01\tabularnewline
			\hline 
			-1.2125e-18 & -2.8615e-16 & -5.6207e-01 & -2.5833e-02 & 1.2375e-02\tabularnewline
			\hline 
			-4.1800e-19 & 4.6199e-17 & 9.1139e-03 & -3.9180e-01 & -4.0392e-01\tabularnewline
			\hline 
			-1.6696e-19 & -4.2405e-18 & 2.8890e-02 & -4.2895e-01 & 4.1673e-01\tabularnewline
			\hline 
			1.2125e-18 & -2.9426e-16 & -5.9797e-01 & -2.7482e-02 & 1.3166e-02\tabularnewline
			\hline 
			3.8779e-19 & 5.0519e-17 & 9.6959e-03 & -4.1683e-01 & -4.2972e-01\tabularnewline
			\hline 
		\end{tabular}
		\par\end{centering}
	\end{subtable}
	\subcaption{Neon}

	
	\par

	\end{centering}
	\protect
	\caption{Constants for combining contracted GTOs for Helium, Beryllium and
	Neon.}
	\label{tab:He_Be_Ne_K}

\end{table}


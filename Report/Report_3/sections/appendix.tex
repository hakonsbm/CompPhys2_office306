\appendix

\section{Program overview}
	\subsection{The Solver}
	The variational monte carlo program can either be used directly as a cpp program, or with a python run.py script included in the folder ./python/ which calls the main cpp program with specific setting, the spesific settings is included in the README file. When running the VMC program it is ran with several parameters specified; number of processors to use, cycles to run, trialfunction to use and what it should do. There are several different options included as what it can do e.g. running several runs searching for te best \(\alpha \) and $\beta$ values, doing a long run writing down positions and several other options. The solver has several subclasses containing trialfunctions, derivatives and determinants. See \cref{fig:schematic} and \cref{fig:classes} for an overview over the program.

	\begin{figure}
		\begin{tikzpicture}[%
		    >=triangle 60,              % Nice arrows; your taste may be different
		    start chain=going below,    % General flow is top-to-bottom
		    node distance=6mm and 45mm, % Global setup of box spacing
		    every join/.style={norm},   % Default linetype for connecting boxes
		    ]
		% ------------------------------------------------- 
		% A few box styles 
		% <on chain> *and* <on grid> reduce the need for manual relative
		% positioning of nodes
		\tikzset{
		  base/.style={draw, on chain, on grid, align=center, minimum height=4ex},
		  proc/.style={base, rectangle, text width=10em},
		  test/.style={base, diamond, aspect=2, text width=5em},
		  term/.style={proc, rounded corners},
		  % coord node style is used for placing corners of connecting lines
		  coord/.style={coordinate, on chain, on grid, node distance=6mm and 45mm},
		  % nmark node style is used for coordinate debugging marks
		  nmark/.style={draw, cyan, circle, font={\sffamily\bfseries}},
		  % -------------------------------------------------
		  % Connector line styles for different parts of the diagram
		  norm/.style={->, draw, lcnorm},
		  free/.style={->, draw, lcfree},
		  cong/.style={->, draw, lccong},
		  it/.style={font={\small\itshape}}
		}
		% -------------------------------------------------
		% Start by placing the nodes
		\node[proc, densely dotted, it] (init) {Initialize solver};
		\node[term, join] (split)      {Split into several threads for multi core};
		\node[term, join] (position)      {Suggest move};
		\node[term, join] (SD) { Compute/update \( |D| \) };
		\node[term, join ] (metro) {Compute Metropolis Ratio};
		\node[test, densely dotted , join ]	(test)	{\(R \ge r\)};
		\node[term]	(new_pos)	{\(\vb{r}^{old} = \vb{r}^{new}\)};
		\node[term, join ]	(energy)	{ Store \(E_L\) };
		\node[test, densely dotted ,join ]	(last)	{Last cycle?};
		\node[term]	(end)	{Collect samples};


		%Setting up the nodes on the side
		\node [term, right=of SD] (trialfunction) {Compute \( \psi_T(\vb{r}) \)};
		\node [term, left=of SD] (quantum) { Compute  Quantumforce};
		\node[term, left=of test] (old_pos) {Keep \(  \vb{r}^{old} \)};
		\node [coord, left=of new_pos] (c1)  {};    
		\node[coord, right=of last]	(around1){};
		\node[coord, right=of around1] (around2) {};
		\node[coord, right=of position]	(around3){};
		\node[coord, right=of around3]	(around4){};


		%Draw new links between boxes
		% \path (SD.south) to node [near start, xshift=1em] {$y$} (quantum);
		\draw [->,lcnorm] (SD.west) -- (quantum);
		\draw [->,lcnorm] (SD.east) -- (trialfunction);
		\draw [->, lcnorm] (quantum.south) -- (metro);
		\draw [->, lcnorm] (trialfunction.south) -- (metro);
		\draw [*->, lccong, , dotted] (test.west) -- (old_pos);
			\path (test.west) to node [ yshift = -1em] {no} (old_pos);
		\draw [*->, lcfree, dotted] (test.south) -- (new_pos);
			\path (test.south) to node [xshift = -1em]{yes} (new_pos);

		\draw [-, lcnorm] (old_pos.south) -- (c1);
		\draw [->, lcnorm] (c1.south) -- (energy);

		\draw[*-, lccong, dotted] (last.east) -- (around2);
			\path (last.east) to node [yshift = -1em] {no} (last);
			\draw[-, lccong, dotted] (around2.east) -- (around4);
			\draw[->, lccong, dotted] (around4) -- (position);

		\draw [*->, lcfree, dotted] (last.south) -- (end);
			\path (last.south) to node [xshift = -1em]{yes} (new_pos);


		\end{tikzpicture}
		\caption{Schematic overview over the workflow of the VMC solver}
		\label{fig:schematic}
	\end{figure}
\subsection{Class structure}
	\begin{figure}
		\begin{tikzpicture}[%
			    >=triangle 60,              % Nice arrows; your taste may be different
			    start chain=going below,    % General flow is top-to-bottom
			    node distance=6mm and 45mm, % Global setup of box spacing
			    every join/.style={norm},   % Default linetype for connecting boxes
			    ]
			% ------------------------------------------------- 
			% A few box styles 
			% <on chain> *and* <on grid> reduce the need for manual relative
			% positioning of nodes
			\tikzset{
			  base/.style={draw, on chain, on grid, align=center, minimum height=4ex},
			  proc/.style={base, rectangle, text width=10em},
			  test/.style={base, diamond, aspect=2, text width=5em},
			  term/.style={proc, rounded corners},
			  % coord node style is used for placing corners of connecting lines
			  coord/.style={coordinate, on chain, on grid, node distance=6mm and 45mm},
			  % nmark node style is used for coordinate debugging marks
			  nmark/.style={draw, cyan, circle, font={\sffamily\bfseries}},
			  % -------------------------------------------------
			  % Connector line styles for different parts of the diagram
			  norm/.style={->, draw, lcnorm},
			  free/.style={->, draw, lcfree},
			  cong/.style={->, draw, lccong},
			  it/.style={font={\small\itshape}}
			}

			%Center column
			\node[term, fill=lcfree!25,  align=center] (solver) {VMCSolver};
			\node[coord]	(blank)	{};
			\node[term] (trialfunction)	{Trialfunction};
				\draw[->, lcnorm]	(solver.south) -- (trialfunction);
			\node[term, join]	(diff)	{He, Be, Ne, H\(_2\) or Be\(_2\)};


			%Sides with lines drawn
			\node[term, right=of blank] (derivatives) {Derivatives};
				\draw[->, lcnorm] (solver.east) -- (derivatives.west);

			\node[term, left=of blank] 	(slater)	{SlaterDeterminant};
				\draw[->, lcnorm] (solver.west) -- (slater.east);

			%Dotted lines between the connected classes
			\draw[-, lcfree, densely dotted] (slater.east) -- (derivatives.west);
			\draw[-, lcfree, densely dotted] (slater.east) -- (trialfunction.west);
			\draw[-, lcfree, densely dotted] (trialfunction.east) -- (derivatives.west);

		\end{tikzpicture}

		\caption{Class and subclass structure used by the program}
		\label{fig:classes}
	\end{figure}

	\subsection{Python Programs}
		There is included several different python programs to help with processing the data, collected with the VMC solver, which is located in the python folder. There is several plotting scripts there and a script to calculate part of the derivatives of the wavefunctions.

\section{Closed expression for noncorrelation Helium trialfunction}

	\subsection{Derivation of local energies, using radial coordinates}
		The local energy of is dependant on the Hamiltonian and the wavefunction describing the system, the Hamiltonian incorporates both a kinetic energy part given by \( \frac{\nabla_i^2}{2} \) for each particle
		and a potential energy part given by \(\frac{Z}{r_i}\) and \(\frac{1}{r_{ij}}\), where \(Z\) is the charge of the center, \(r_i\) is the distance for electron \(i\) to the atom center and \(r_{ij}\) is the distance between electron \(l\) and \(m\). Then the local energy is given by the following:

		\begin{align}
			E_L &= \sum_{i,i<j}{\frac{1}{ \Psi_T(\vb{r_i} , \vb{r_{ij}}) } \hat{H} \Psi_T(\vb{r_i} , \vb{r_{ij}})}
			\\
			&=	\sum_{i,i<j}\frac{1}{ \Psi_T(\vb{r_i} , \vb{r_{ij}}) } \left( - \frac{\nabla_i^2}{2} -\frac{Z}{r_i}  -  \frac{Z}{r_j} +  \frac{1}{r_{ij} }  \right) \Psi_T(\vb{r_i} , \vb{r_{ij}})
			\\
			&= \sum_{i,i<j}{-\frac{1}{2\Psi_T} \left(\nabla_i^2 \Psi_T  \right)  -\frac{Z}{r_i}  -  \frac{Z}{r_j} +  \frac{1}{r_{ij} }}
		\end{align}

		Let us change derivation variables:

		\begin{align}
			-\frac{1}{2\Psi_T} \left(\nabla_i^2 \Psi_T  \right) &= \sum_{m=1}^{3}{-\frac{1}{2\Psi_T} \left( \pdv[2]{\Psi_T}{x_m} \right)_i}
			\\
			&= \sum_{m=1}^{3}{-\frac{1}{2\Psi_T} \left( \pdv{}{x_m} \left( \pdv{\Psi_T}{r_i}\pdv{r_i}{x_m} \right) \right)_i}
			\intertext{Since \(r_i = \left( x_1^2 + x_2^2 + x_3^2 \right)^{1/2}\) then \( \pdv{r_i}{x_m} = \pdv{\left( x_1^2 + x_2^2 + x_3^2 \right)^{1/2}}{x_m} =\frac{x_m}{r_i} \)}
			&= \sum_{m=1}^{3}{-\frac{1}{2\Psi_T} \left( \pdv{}{x_m} \left( \pdv{\Psi_T}{r_i}\frac{x_m}{r_i} \right) \right)_i}
			\\
			&= \sum_{m=1}^{3}{-\frac{1}{2\Psi_T} \left( \pdv{\Psi_T}{x_m}{r_i}\frac{x_m}{r_i} + \pdv{\Psi_T}{r_i} \pdv{}{x_m} \left(\frac{x_m}{r_i} \right) \right)_i}
			\intertext{ The term \( \pdv{}{x_m} \left(\frac{x_m}{r_i} \right) \) becomes for the different values for \(m\),  \(\pdv{}{x_1}  \left( \frac{x_1}{\left( x_1^2 + x_2^2 + x_3^2 \right)^{1/2}} \right) = \frac{x_2^2 + x_3^2}{r_i^3}\) so all the values for \(m\) term it should sum up to \( \frac{ 2 (x_1^2 + x_2^2 + x_3^2) }{ r_i^3 } \) }
			&= -\frac{1}{2\Psi_T} \left( \pdv[2]{\Psi_T}{r_i}\frac{x_1^2 + x_2^2 + x_3^2}{r^2_i} + \pdv{\Psi_T}{r_i} \frac{ 2 (x_1^2 + x_2^2 + x_3^2) }{ r_i^3 } \right)_i
			\\
			&= -\frac{1}{2\Psi_T} \left( \pdv[2]{\Psi_T}{r_i} + \pdv{\Psi_T}{r_i} \frac{ 2 }{ r_i } \right)
		\end{align}
		Then the local energy becomes:
		\begin{align}
			E_L = \sum_{i,i<j}{  -\frac{1}{2\Psi_T} \left( \pdv[2]{\Psi_T}{r_i} + \pdv{\Psi_T}{r_i} \frac{ 2 }{ r_i } \right)  -\frac{Z}{r_i}  -  \frac{Z}{r_j} +  \frac{1}{r_{ij} }} \label{eq:localEnergy}
		\end{align}

		We can apply this to the simple helium trialfunction with no electronic interaction to obtain the local energy.

		\subsubsection{Helium: Simple trialfunction}
		The simple version of the trial function is only dependant on one parameter \( \alpha \) and does not take into account interaction between the two electrons, it is of the form
		\[ \Psi_T (\vb{r_1}, \vb{r_2}) = \exp{ -\alpha (r_1 + r_2) } \]Let us set this trialfunction into the equation for the local energy \eqref{eq:localEnergy}.
		\begin{align}
			E_L &= \sum_{i,i<j}{  -\frac{1}{2\Psi_T} \left( \pdv[2]{e^{-\alpha (r_i + r_j)}}{r_i} + \pdv{e^{-\alpha (r_i + r_j)}}{r_i} \frac{ 2 }{ r_i } \right)  -\frac{Z}{r_i}  -  \frac{Z}{r_j} +  \frac{1}{r_{ij} }}
			\\
			E_L &= -\frac{1}{2\Psi_T} \sum_{i=1}^2{ \left( \alpha^2 -\alpha \frac{ 2 }{ r_i } \right) \Psi_T  -\frac{Z}{r_i} +  \frac{1}{r_{ij} } }
			\\
			E_L &= -\alpha^2 + (\alpha-Z) \left( \frac{1}{r_1} + \frac{1}{r_2} \right) + \frac{1}{r_{12}} \label{eq:heliumLocalEnergy}
		\end{align}

	\section{GTO tables}
	\label{sec:GTO_app}
		\begin{table}
			\begin{centering}
			\begin{subtable}{1.0\linewidth}
				\begin{centering}
				\begin{tabular}{|c|}
					\hline 
					1s\tabularnewline
					\hline 
					0.4579\tabularnewline
					\hline 
					0.6573\tabularnewline
					\hline 
				\end{tabular}
				\par\end{centering}
			\end{subtable}
			\subcaption{Helium}

			\begin{subtable}{1.0\linewidth}
				\begin{centering}
				\begin{tabular}{|c|c|}
					\hline 
					1s & 2s\tabularnewline
					\hline 
					-9.9281e-01 & -2.1571e-01\tabularnewline
					\hline 
					-7.6425e-02 & 2.2934e-01\tabularnewline
					\hline 
					2.8727e-02 & 8.2235e-01\tabularnewline
					\hline 
					1.2898e-16 & 5.1721e-16\tabularnewline
					\hline 
					-2.3257e-19 & 4.5670e-18\tabularnewline
					\hline 
					5.6097e-19 & -1.1040e-17\tabularnewline
					\hline 
					1.2016e-16  & 8.5306e-16\tabularnewline
					\hline 
					-4.6874e-19 & 7.0721e-18\tabularnewline
					\hline 
					1.1319e-18 & -1.7060e-17\tabularnewline
					\hline 
				\end{tabular}
				\par\end{centering}
			\end{subtable}
			\subcaption{Beryllium}

			\par
			\begin{subtable}{1.0\linewidth}
				\begin{centering}
				\begin{tabular}{|c|c|c|c|c|}
					\hline 
					1s & 2s & $2p_{x}$  & $2p_{y}$ & $2p_{z}$\tabularnewline
					\hline 
					-9.8077e-01 & -2.6062e-01 & 1.1596e-16 & -8.3716e-18 & -1.9554e-17\tabularnewline
					\hline 
					-9.3714e-02 & 2.5858e-01 & -2.0106e-16 & -9.7173e-17 & -7.3738e-17\tabularnewline
					\hline 
					2.2863e-02 & 8.1619e-01 & -3.2361e-16 & 1.3237e-16 & 1.5789e-16\tabularnewline
					\hline 
					-9.9519e-19  & -5.6186e-18 & 2.7155e-02 & -4.0320e-01 & 3.9171e-01\tabularnewline
					\hline 
					-1.2125e-18 & -2.8615e-16 & -5.6207e-01 & -2.5833e-02 & 1.2375e-02\tabularnewline
					\hline 
					-4.1800e-19 & 4.6199e-17 & 9.1139e-03 & -3.9180e-01 & -4.0392e-01\tabularnewline
					\hline 
					-1.6696e-19 & -4.2405e-18 & 2.8890e-02 & -4.2895e-01 & 4.1673e-01\tabularnewline
					\hline 
					1.2125e-18 & -2.9426e-16 & -5.9797e-01 & -2.7482e-02 & 1.3166e-02\tabularnewline
					\hline 
					3.8779e-19 & 5.0519e-17 & 9.6959e-03 & -4.1683e-01 & -4.2972e-01\tabularnewline
					\hline 
				\end{tabular}
				\par\end{centering}
			\end{subtable}
			\subcaption{Neon}

			
			\par

			\end{centering}
			\protect
			\caption{Constants for combining contracted GTOs for Helium, Beryllium and
			Neon.}
			\label{tab:He_Be_Ne_K}

		\end{table}

\subsection{Atoms and Molecules}
	%General atoms stuff that applies to all the different atoms
	The dimensionless hamiltonian for the case of electrons around a nucleus is given by 

	\begin{align}
		\hat{H} &= \sum_{i = 1}^N - \frac{\nabla^2_i}{2} - \frac{Z}{r_i} + \sum_{i < j}\frac{1}{r_{ij}} \label{eq:hamiltonian}
	\end{align}

	where \(r_i\) is the distance from electron \(i\) to the nucleus, \(Z\) is the nucleus charge, and \(r_{ij} = |\vb{r}_i - \vb{r}_j|\).
	The kinetic energy for electron \(i\) is represented by \( - \frac{\nabla^2_i}{2} \), \(- \frac{Z}{r_i}\) the potential energy with respect to the nucleus and \( \frac{1}{r_{ij}} \) the repulsive energy between the electrons \(i\) and \( j\).

	In the VMC calculation the local energy, \(E_L\) is a useful quantity and needs to be finite at all points to be normalizable. So by looking at the limits where the \(E_L\) diverges we can quess the form the wavefunctions should follow.

	\begin{align}
		E_L(r_i,r_{ij}) &= \frac{1}{\Psi_T} \hat{H} \Psi_T
	\end{align}

	In the cases where \(r_i \rightarrow 0\) or \( r_{ij} \rightarrow 0\) we need to make sure that the local energy does not diverge.


		\begin{align}
			\lim_{r_i\rightarrow 0} {E_L(r_i,r_{ij}) } &= \frac{1}{R_i(r_i)} \left( - \frac{1}{2}\pdv[2]{}{x_k} - \frac{Z}{r_i} \right) R_i(r_i) + G(r_i, r_{ij})
			\\
			\lim_{r_i\rightarrow 0} {E_L(r_i,r_{ij}) } &= \frac{1}{R_i(r_i)} \left( - \frac{1}{2}\pdv[2]{}{r_k} - \frac{1}{r_i}\pdv{}{r_i}	 -	 \frac{Z}{r_i} \right) R_i(r_i) + G(r_i, r_{ij})
			\intertext{Given a well behaved wavefunction \( \frac{1}{2}\pdv[2]{}{r_k} \) is finite.}
			\lim_{r_i\rightarrow 0} {E_L(r_i,r_{ij}) } &= 
			\frac{1}{R_i(r_i)} \left( - \frac{1}{r_i}\pdv{}{r_i}	 -	 \frac{Z}{r_i} \right) R_i(r_i)
		\end{align}

		This is finite given when the following differential equation is fulfilled.

		\begin{align}
			\frac{1}{R_i(r_i)} \pdv{}{r_i} R_i(r_i)	=  -Z  \qquad{ \text{ With solution }}  \qquad R_i(r_i) = A e^{-Z}
		\end{align}

		A similar calculation applies for \(r_{12} \rightarrow 0\) and a trialfunction of the form   \[\Psi_T(r_i,r_j,r_{ij}) = e^{ -\alpha \sum_{N}  r_i} \prod^N_{i < j}e^{\beta r_{ij}} = e^{ -\alpha \sum_{N}  r_i} \prod^N_{i < j}e^{\frac{a r_ij}{1 + \beta r_{ij}}} \] should fulfill the condition that the local energy is finite.


	\subsubsection{Helium atom}
		The hamiltonian for the helium atom is given by equation \eqref{eq:hamiltonian}

	\subsubsection{Beryllium atom}

		It is fairly simple to extend the calculational machinery of Variational
		Monte Carlo to other systems than the Helium atom. To show this we
		want to perform calculations on the beryllium atom. As beryllium has
		four electrons compared to the 2 of helium, we need to calculate a
		Slater determinant. However the computation of the Slater determinant
		can be simplified for beryllium. Sticking to hydrogen-like wave functions,
		we can write the trial wave function for beryllium as
		\begin{equation}
			\psi_{T}({\bf r_{1}},{\bf r_{2}},{\bf r_{3}},{\bf r_{4}})=Det\left(\phi_{1}({\bf r_{1}}),\phi_{2}({\bf r_{2}}),\phi_{3}({\bf r_{3}}),\phi_{4}({\bf r_{4}})\right)\prod_{i<j}^{4}\exp{\left(\frac{r_{ij}}{2(1+\beta r_{ij})}\right)},
			\label{eq:BerylliumTrialFunction}
		\end{equation}
		where $Det$ is a Slater determinant and the single-particle wave
		functions are the hydrogen wave functions for the 1s and 2s orbitals.
		With the variational ansatz these are
		\begin{align}
			\phi_{1s}({\bf r_{i}})=e^{-\alpha r_{i}},
		\end{align}
		and
		\begin{align}
			\phi_{2s}({\bf r_{i}})=\left(1-\alpha r_{i}/2\right)e^{-\alpha r_{i}/2}.
		\end{align}
		The Slater determinant is calculated using these ansatzes.

		Furthermore, for the Jastrow factor,
		\begin{align}
			\Psi_{C}=\prod_{i<j}g(r_{ij})=\exp{\sum_{i<j}\frac{ar_{ij}}{1+\beta r_{ij}}},
		\end{align}
		we need to take into account the spins of the electrons. We fix electrons
		1 and 2 to have spin up, and electron 3 and 4 to have spin down. This
		means that when the electrons have equal spins we get a factor
		\begin{align}
			a=\frac{1}{4},
		\end{align}
		and if they have opposite spins we get a factor
		\begin{align}
			a=\frac{1}{2}.
		\end{align}

	\subsubsection{Neon atom}

		Wishing to extend our variational Monte Carlo machinery further we implement Neon. Neon has ten electrons, so it is a big jump from Helium and Beryllium. Therefore we also have to implement a better way to handle the Slater determinant than we did in the previous project. The trial wave function for Neon can be written as
		\begin{equation}
		   \psi_{T}({\bf r_1},{\bf r_2}, \dots,{\bf r_{10}}) =
		   Det\left(\phi_{1}({\bf r_1}),\phi_{2}({\bf r_2}),
		   \dots,\phi_{10}({\bf r_{10}})\right)
		   \prod_{i<j}^{10}\exp{\left(\frac{r_{ij}}{2(1+\beta r_{ij})}\right)},
		   \label{eq:NeonTrialFunction}
		\end{equation}
		Now we need to include the $2p$ wave function as well. It is given as
		\begin{equation}
			\phi_{2p}({\bf r_i}) = \alpha {\bf r_i}e^{-\alpha r_i/2}.
		\end{equation}
		where $ {\bf r_i} = \sqrt{r_{i_x}^2+r_{i_y}^2+r_{i_z}^2}$.
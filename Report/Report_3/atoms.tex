\subsection{Atoms and Molecules}
	Did we get here??
\subsection{Helium atom}

		The dimensionless hamiltonian for the Helium atom consists of a kinetic energy part and a potential energy part and is given by

		\begin{align}
			\hat{H} &= -\frac{\nabla^2_1}{2} - \frac{\nabla^2_2}{2} - \frac{Z}{r_1} - \frac{Z}{r_2} + \frac{1}{r_{12}}
		\end{align}

		For the energy to be finite the wavefunction must be constructed so the local energy is finite at all points.

		\begin{align}
			E_L(r_i,r_{12}) &= \frac{1}{\Psi_T} \hat{H} \Psi_T
		\end{align}

		If we consider the case where \(r_i \rightarrow 0\) then the \(- \frac{\nabla^2_i}{2} - \frac{Z}{r_i} \) terms of the hamiltonian could cause the energy to blow up, so we need to make sure that those terms stay finite in the limit.

		\begin{align}
			\lim_{r_i\rightarrow 0} {E_L(r_1,r_{12}) } &= \frac{1}{R_i(r_i)} \left( - \frac{1}{2}\pdv[2]{}{x_k} - \frac{Z}{r_i} \right) R_i(r_i) + G(r_i, r_{ij})
			\\
			\lim_{r_i\rightarrow 0} {E_L(r_1,r_{12}) } &= \frac{1}{R_i(r_i)} \left( - \frac{1}{2}\pdv[2]{}{r_k} - \frac{1}{r_i}\pdv{}{r_i}	 -	 \frac{Z}{r_i} \right) R_i(r_i) + G(r_i, r_{ij})
			\intertext{ Derivatives of the wavefunction does not diverge since the wavefunction is finite at all points. so the following terms dominate when the particles approach the center. }
			\lim_{r_i\rightarrow 0} {E_L(r_1,r_{12}) } &= \frac{1}{R_i(r_i)} \left( - \frac{1}{r_i}\pdv{}{r_i}	 -	 \frac{Z}{r_i} \right) R_i(r_i)
			\end{align}

		\begin{align}
			 \frac{1}{R_i(r_i)} \pdv{}{r_i} R_i(r_i)	=  -Z  \qquad{ \text{ With solution }}  \qquad R_i(r_i) = A e^{-Z}
		\end{align}

		A similar calculation applies for \(r_{12} \rightarrow 0\) and a trialfunction of the form   \[\Psi_T(r_1,r_2,r_{12}) = e^{-\alpha (r_1 + r_2)} e^{\beta r_{12}} \] should fulfill the condition that the wavefunction is finite everywhere.

	\subsection{Beryllium atom}

		It is fairly simple to extend the calculational machinery of Variational
		Monte Carlo to other systems than the Helium atom. To show this we
		want to perform calculations on the beryllium atom. As beryllium has
		four electrons compared to the 2 of helium, we need to calculate a
		Slater determinant. However the computation of the Slater determinant
		can be simplified for beryllium. Sticking to hydrogen-like wave functions,
		we can write the trial wave function for beryllium as
		\begin{equation}
			\psi_{T}({\bf r_{1}},{\bf r_{2}},{\bf r_{3}},{\bf r_{4}})=Det\left(\phi_{1}({\bf r_{1}}),\phi_{2}({\bf r_{2}}),\phi_{3}({\bf r_{3}}),\phi_{4}({\bf r_{4}})\right)\prod_{i<j}^{4}\exp{\left(\frac{r_{ij}}{2(1+\beta r_{ij})}\right)},
			\label{eq:BerylliumTrialFunction}
		\end{equation}
		where $Det$ is a Slater determinant and the single-particle wave
		functions are the hydrogen wave functions for the 1s and 2s orbitals.
		With the variational ansatz these are
		\begin{align}
			\phi_{1s}({\bf r_{i}})=e^{-\alpha r_{i}},
		\end{align}
		and
		\begin{align}
			\phi_{2s}({\bf r_{i}})=\left(1-\alpha r_{i}/2\right)e^{-\alpha r_{i}/2}.
		\end{align}
		The Slater determinant is calculated using these ansatzes.

		Furthermore, for the Jastrow factor,
		\begin{align}
			\Psi_{C}=\prod_{i<j}g(r_{ij})=\exp{\sum_{i<j}\frac{ar_{ij}}{1+\beta r_{ij}}},
		\end{align}
		we need to take into account the spins of the electrons. We fix electrons
		1 and 2 to have spin up, and electron 3 and 4 to have spin down. This
		means that when the electrons have equal spins we get a factor
		\begin{align}
			a=\frac{1}{4},
		\end{align}
		and if they have opposite spins we get a factor
		\begin{align}
			a=\frac{1}{2}.
		\end{align}

	\subsection{Neon atom}

		Wishing to extend our variational Monte Carlo machinery further we implement Neon. Neon has ten electrons, so it is a big jump from Helium and Beryllium. Therefore we also have to implement a better way to handle the Slater determinant than we did in the previous project. The trial wave function for Neon can be written as
		\begin{equation}
		   \psi_{T}({\bf r_1},{\bf r_2}, \dots,{\bf r_{10}}) =
		   Det\left(\phi_{1}({\bf r_1}),\phi_{2}({\bf r_2}),
		   \dots,\phi_{10}({\bf r_{10}})\right)
		   \prod_{i<j}^{10}\exp{\left(\frac{r_{ij}}{2(1+\beta r_{ij})}\right)},
		   \label{eq:NeonTrialFunction}
		\end{equation}
		Now we need to include the $2p$ wave function as well. It is given as
		\begin{equation}
			\phi_{2p}({\bf r_i}) = \alpha {\bf r_i}e^{-\alpha r_i/2}.
		\end{equation}
		where $ {\bf r_i} = \sqrt{r_{i_x}^2+r_{i_y}^2+r_{i_z}^2}$.
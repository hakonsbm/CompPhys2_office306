\documentclass[11pt]{article}
\usepackage{verbatim}
\usepackage{listings}
\usepackage{graphicx}
\usepackage{a4wide}
\usepackage{color}
\usepackage{amsmath}
\usepackage{amssymb}
\usepackage[dvips]{epsfig}
\usepackage[T1]{fontenc}
\usepackage{cite} % [2,3,4] --> [2--4]
\usepackage{shadow}
\usepackage{hyperref}
\usepackage{physics}
\usepackage{url}
\usepackage{tikz}
\usepackage{subcaption}
\usepackage[utf8]{inputenc}
\usepackage{booktabs} % Allows the use of \toprule, \midrule and \bottomrule in tables





\usetikzlibrary{arrows, shapes}

\setcounter{tocdepth}{2}

\lstset{language=c++}
\lstset{alsolanguage=[90]Fortran}
\lstset{basicstyle=\small}
\lstset{backgroundcolor=\color{white}}
\lstset{frame=single}
\lstset{stringstyle=\ttfamily}
\lstset{keywordstyle=\color{red}\bfseries}
\lstset{commentstyle=\itshape\color{blue}}
\lstset{showspaces=false}
\lstset{showstringspaces=false}
\lstset{showtabs=false}
\lstset{breaklines}

\title{ FYS-4411: Computational Physics II \\ Project 1 }
\author{Gullik Vetvik Killie\\
		Håkon Sebastian Bakke Mørk\\
		Jose Emilio Ruiz Navarro
		}

\begin{document}

\maketitle

\abstract{}

\subsection{Efficient calculation of derivatives}
	Calculating the derivatives involved in the VMC calculation numerically is slow in that they entail several calls to the wavefunctions in addition to introducing an extra numerical error. Here we will show how we have found divided up the derivatives and found analytic expressions for all the parts.

	The trialfunction can be factorized as
	\begin{align}
	\Psi_T(\vb{x}) &= \Psi_{D} \Psi_C= |D_\uparrow| |D_\downarrow| \Psi_C \label{eq:factorization}
	\end{align}

	where \(D_\uparrow\), \(D_\downarrow\) and \(\Psi_C\) is the spin up and down part of the Slater determinant and the Jastrow factor respectively.

	\subsubsection{Kinetic Energy}
		From the Hamiltonian the expectation value of kinetic energy for each electron is given by

		\begin{align}
			K_i &= - \frac{1}{2} \frac{\nabla^2_i \Psi}{\Psi}
		\end{align}

			Using the factorization of of the trialfunction from \eqref{eq:factorization} we can calculated the ratio needed for the kinetic energy.
		\begin{align}
			\frac{1}{\Psi_T}\pdv[2]{\Psi_T}{x_k} &= \frac{1}{\Psi_D\Psi_C} \pdv[2]{(\Psi_D\Psi_C)}{x_k} = \frac{1}{\Psi_D\Psi_C} \pdv{}{x_k} \left( \pdv{\Psi_D}{x_k} \Psi_C +\Psi_D \pdv{\Psi_C}{x_k} \right)
			\\
			&= \frac{ 1 }{\Psi_D\Psi_C} \left( \pdv[2]{\Psi_D}{x_k} \Psi_C   + 2 \pdv{ \Psi_D }{x_k}\pdv{ \Psi_C }{x_k} + \Psi_D\pdv[2]{\Psi_C}{x_k} \right)
			\\
			&= \frac{1}{\Psi_D}\pdv[2]{\Psi_D}{x_k}  + 2 \frac{1}{\Psi_D} \pdv{ \Psi_D }{x_k} \cdot \frac{1}{\Psi_C}\pdv{ \Psi_C }{x_k} +  \frac{1}{\Psi_C}\pdv[2]{\Psi_C}{x_k} \label{eq:laplacianIntermediate}
		\end{align}

		Since the Slater determinant part of the trialfunction is seperable into a spin up and down part we can simplify it further.

		\begin{align}
			\frac{1}{\Psi_D}\pdv[2]{\Psi_D}{x_k} &= \frac{1}{|D_\uparrow| |D_\downarrow|} \pdv[2]{ |D_\uparrow| |D_\downarrow| }{x_k}
			= \frac{1}{|D_\uparrow|} \pdv[2]{|D_\uparrow|}{x_k} + \frac{1}{|D_\downarrow|} \pdv[2]{|D_\downarrow|}{x_k} \label{eq:lapplacianSlaterRatio}
			\\
			\frac{1}{\Psi_D} \pdv{ \Psi_D }{x_k}  &=  \frac{1}{|D_\uparrow| |D_\downarrow|} \pdv{ |D_\uparrow| |D_\downarrow| }{x_k}
			= \frac{1}{|D_\uparrow|} \pdv{|D_\uparrow|}{x_k} + \frac{1}{|D_\downarrow|} \pdv{|D_\downarrow|}{x_k} \label{eq:gradianSlaterRatio}
		\end{align}

		Inserting equations \eqref{eq:gradianSlaterRatio} and \eqref{eq:lapplacianSlaterRatio} into \eqref{eq:laplacianIntermediate} we get

		\begin{align}
			\frac{\nabla^2 \Psi_T}{\Psi_T} &= \frac{\nabla^2 |D_\uparrow|}{|D_\uparrow|} + \frac{\nabla^2 |D_\downarrow|}{|D_\downarrow|} + 2 \left( \frac{\nabla |D_\uparrow|}{|D_\uparrow|} + \frac{\nabla |D_\downarrow|}{|D_\downarrow|} \right) \cdot \frac{\nabla\Psi_C}{\Psi_C} +  \frac{\nabla^2\Psi_C}{\Psi_C}
		\end{align}

		Now we have \(4\) different types of ratios we need to find an expression for, \( \frac{\nabla^2 |D|}{|D|} \) , \(\frac{\nabla |D|}{|D|} \), \( \frac{\nabla^2\Psi_C}{\Psi_C} \) and \( \frac{\nabla\Psi_C}{\Psi_C} \).

\end{document}